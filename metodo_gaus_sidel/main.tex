O método de Gauss-Seidel é uma variação do método de Jacobi para a solução de sistemas lineares da forma

        \[
        A \mathbf{x} = \mathbf{b},
        \]
        
        \noindent em que a matriz $A$ é quadrada e não possui elementos nulos em sua diagonal. A principal diferença em relação ao método de Jacobi está no processo de iteração: a cada novo valor calculado $x_i^{(k+1)}$, este já é imediatamente utilizado no cálculo dos demais valores da mesma iteração.
        
        \medskip
        
        Considere o sistema linear de $n$ equações:
        
        \[
        A =
        \begin{bmatrix}
        a_{11} & a_{12} & \cdots & a_{1n} \\
        a_{21} & a_{22} & \cdots & a_{2n} \\
        \vdots & \vdots & \ddots & \vdots \\
        a_{n1} & a_{n2} & \cdots & a_{nn}
        \end{bmatrix},
        \quad
        \mathbf{x} =
        \begin{bmatrix}
        x_{1} \\
        x_{2} \\
        \vdots \\
        x_{n}
        \end{bmatrix},
        \quad
        \mathbf{b} =
        \begin{bmatrix}
        b_{1} \\
        b_{2} \\
        \vdots \\
        b_{n}
        \end{bmatrix}.
        \]
        
        Decompomos a matriz $A$ em três partes: a diagonal $D$, a parte triangular inferior $L$ e a parte triangular superior $U$, de modo que
        
        \[
        A = D + L + U.
        \]
        
        Assim, o sistema linear pode ser reescrito como:
        
        \begin{equation}
        (D + L + U)\mathbf{x} = \mathbf{b}.
        \label{eq:Sistema_GS}
        \end{equation}
        
        Isolando $\mathbf{x}$ em \eqref{eq:Sistema_GS}, obtemos a forma iterativa:
        
        \begin{equation}
        \mathbf{x}^{(k+1)} = (D + L)^{-1}\big(\mathbf{b} - U \mathbf{x}^{(k)}\big).
        \label{eq:Metodo_GS_Matricial}
        \end{equation}
        
        De forma equivalente, elemento a elemento:
        
        \begin{equation}
        x_i^{(k+1)} = \frac{1}{a_{ii}}
        \left( b_i - \sum_{j=1}^{i-1} a_{ij} x_j^{(k+1)} - \sum_{j=i+1}^n a_{ij} x_j^{(k)} \right),
        \quad i = 1, 2, \ldots, n.
        \label{eq:Metodo_GS_Algebrico}
        \end{equation}
        
        \noindent Ou seja, para calcular $x_i^{(k+1)}$, utilizamos imediatamente os novos valores já computados $x_1^{(k+1)}, x_2^{(k+1)}, \ldots, x_{i-1}^{(k+1)}$, em conjunto com os valores antigos $x_{i+1}^{(k)}, \ldots, x_{n}^{(k)}$.
        
        \medskip
        
        A convergência do método de Gauss-Seidel não é garantida para qualquer matriz $A$. Em geral, a convergência é assegurada se a matriz $A$ for estritamente diagonalmente dominante ou simétrica definida positiva. Nessas condições, a sequência de aproximações $\{\mathbf{x}^{(k)}\}$ tende para a solução exata $\mathbf{x}$ do sistema linear.
        
        Como o método é iterativo, define-se um critério de parada baseado em um erro. Uma forma comum é utilizar o erro absoluto entre iterações consecutivas:
        
        \[
        \| \mathbf{x}^{(k+1)} - \mathbf{x}^{(k)} \| < \varepsilon,
        \]
        
        onde $\varepsilon$ é a tolerância previamente estabelecida. Outra possibilidade é calcular o erro relativo, dado por
        
        \[
        \frac{\| \mathbf{x}^{(k+1)} - \mathbf{x}^{(k)} \|}{\| \mathbf{x}^{(k+1)} \|} < \varepsilon,
        \]
        
        o que garante que a precisão desejada não dependa da escala dos valores de $\mathbf{x}$.
        
        Na prática, o processo iterativo é interrompido quando o erro calculado for menor que a tolerância definida, assegurando que a solução aproximada $\mathbf{x}^{(k+1)}$ esteja suficientemente próxima da solução exata.